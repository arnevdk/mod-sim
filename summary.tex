\documentclass{article}
\usepackage[utf8]{inputenc}
\usepackage[dutch]{babel}
\usepackage{amsmath}
\usepackage{algorithm}
\usepackage[noend]{algpseudocode}
\usepackage{bbold}

\title{Samenvatting [G0Q57A] - Modellering en simulatie}
\author{Arne Van Den Kerchove}

\newtheorem{mydef}{Definitie}

\newcommand{\norm}[1]{\left\lVert#1\right\rVert}

\setlength\parindent{0pt}

\begin{document}
	\maketitle
	
	\tableofcontents
	
	\pagebreak
	
	\section{Modellen en simulaties}
	\section{Numerieke lineaire algebra en toepassingen}
	\subsection{QR-factorisatie}
	
	
	\begin{mydef}
		De volle QR-factorisatie van de matrix $A$ wordt gegeven door
		$$	A=QR  $$
		met $q$ een $m \times m$ orthogonale matrix en $R$ een $m \times n $ bovendriehoeksmatrix.
	\end{mydef}

	\subsubsection{Gram-Schmidt orthogonalisatie}
	

	\begin{algorithm}[!ht]
		\caption{Gram-Schmidt-algoritme}
		\begin{algorithmic}[1]
			\Procedure{QRGramSchmidt}{}
				\For{$j=1$ to $n$}
					\State $v_j = a_j$
					\For{$i=1$ to $j-1$}
						\State $r_{ij} = q_i^T a_j$
						\State $v_j = v_j - r_{ij} q_i$
					\EndFor
					\State $r_{jj} = \norm{v_j}_2$
					\State $q_j = v_j/r_{jj}$
				\EndFor 
			\EndProcedure
		\end{algorithmic}
	\end{algorithm}

	\textbf{Complexiteit:} $\mathcal{O}(2mn^2)$\\
	\textbf{Stabiliteit:} niet stabiel
	
	\subsubsection{QR-factorisatie met Givens-rotaties}
	
	\begin{mydef}
		Een Givens-rotatie is een $m \times m$ orthogonale matrix van de vorm
		$$
		G_{ij} = 
		\begin{bmatrix}
			c & -s \\
			s &  c  
		\end{bmatrix}
		$$
		met
		$$ c^2 + s^2 = 1 $$
	\end{mydef}

	Om een Givens-rotatie op te stellen die plaats $(j,k)$ $0$ maakt in matrix $A$, kies dan een element in dezelfde kolom (bv. het element boven $(j,k)$) op oplaats $(i,k)$ en maak $G_{ij}$ met
	$$ c = \frac{a_{ik}}{\sqrt{a_{ik}^2 + a_{jk}^2}} \text{ en } s = \frac{a_{jk}}{\sqrt{a_{ik}^2 + a_{jk}^2}}$$
	
	\pagebreak
	
	\begin{algorithm}[!ht]
		\caption{Givens-rotatie-algoritme}
		\begin{algorithmic}[1]
			\Procedure{QRGivens}{}
				\State $Q=\mathbb{1}$
				\State $R=A$
				\For{$j=1$ to $n$}
					\For{$i=m$ to $j+1$}
						\State $c = \frac{r_{i-1,j}}{\sqrt{r_{i-1,j}^2 + r_{i,j}^2}}$
						\State $s = \frac{r_{i,j}}{\sqrt{r_{i-1,j}^2 + r_{i,j}^2}}$
						\State $r_{i,j} = 0$
						\State $r_{i-1,j} = \sqrt{r_{i-1,j}^2 + r_{i,j}^2}$
						\For{$k=j+1$ to $n$}
							\State 
							$$
								\begin{bmatrix}
									r_{i-1,k} \\
									r_{ik}
								\end{bmatrix}
								=
								\begin{bmatrix}
									c & s \\
									-s & s
								\end{bmatrix}
								\begin{bmatrix}
									r_{i-1,k} \\
									r_{i,k}
								\end{bmatrix}
							$$
						\EndFor
						\For{$k=1$ to $m$}
							\State
							$$
								\begin{bmatrix}
									q_{k,i-1} & q_{ki}
								\end{bmatrix}
								=
								\begin{bmatrix}
									q_{k,i-1} & q_{ki}
								\end{bmatrix}
								\begin{bmatrix}
									c & s \\
									-s & s
								\end{bmatrix}
							$$
						\EndFor
					\EndFor
				\EndFor
			\EndProcedure
		\end{algorithmic}
	\end{algorithm}

	\textbf{Complexiteit:} $\mathcal{O}(3mn^2 - n^3)$ \\
	\textbf{Stabiliteit:} stabiel


	
	\subsubsection{QR-factorisatie met kolompivotering}
	Indien $A$ niet van volle rang is, is het voor de stabiliteit beter om kolompivotering toe te passen. In stap $j$ van het QR-algoritme met Givens-rotaties verwisselen we kolom $j$ met de kolom $p$ waarvan de 2-norm het grootst is.\\
		
	\textbf{Complexiteit:} $\mathcal{O}(3mn^2 - n^3)$ \\
	\textbf{Stabiliteit:} stabieler voor rang-deficiënte matrices
	
	
	\subsection{Kleinste-Kwadratenbenadering}
	Om de coëfficiënten te bepalen wordt een Vandermondematrix $A$ opgesteld. 
	De te minimaliseren fout bij KK-benadering wordt gegeven door
	$$
	\min_{x \in \mathbb{R}^n} \norm{b-Ax}_2 
	= \min_{x \in \mathbb{R}^n} \sqrt{\sum_{i=1}^{m}(b_i - \sum_{j=1}^{n}a_{i,j}x_j)^2}
	$$
	met $r=b-Ax$ de \textit{residuvector}.
	
	Dit probleem kan opgelost worden door $x$ te bepalen in 
	$$
	A^TAx = A^Tb
	$$
	De Vandermondematrix $A$ is slecht geconditioneerd. We zoeken dus andere manieren om het KK-probleem op te lossen.\\
	
	\subsubsection{Oplossing met QR-ontbinding}
	
	Indien de QR-factorisatie van $A$ bekend is, kan deze gebruikt worden om een oplossing voor het KK-probleem te vinden:
	$$
	\min_{x \in \mathbb{R}^n} \norm{b-Ax}_2  
	= \min_{x \in \mathbb{R}^n} \norm{b-QRx}_2
	= \min_{x \in \mathbb{R}^n} \norm{Q^Tb-RAx}_2 
	$$
	Aangezien vermenigvuldiging vooraan met een orthogonale matrix de norm behoudt. \\	
	De  vector $Q^Tb=c$ kan opgesplitst worden in  de volgende componenten:
	$
	\begin{bmatrix}
		c_1\\
		c_2
	\end{bmatrix}
	$
	met $c_1 \in \mathbb{R}^n$ en $c_2 \in \mathbb{R}^{m-n}$\\
	Hieruit volgt:
	$$
	\min_{x \in \mathbb{R}^n} \norm{b-Ax}_2 
	= 	\min_{x \in \mathbb{R}^n} \norm{
		\begin{bmatrix}
		c_1\\
		c_2
		\end{bmatrix}
		-
		\begin{bmatrix}
			\hat{R}\\
			0
		\end{bmatrix}
		x}_2 
	$$
	Volgens de stelling van Pythagoras geldt:
	$$
	\min_{x \in \mathbb{R}^n} \norm{b-Ax}_2^2
	= \min_{x \in \mathbb{R}^n} (\norm{c_1-\hat{R}x}_2^2 + \norm{c_2}_2^2)
	$$
	De vector $x$ met coëfficiënten met minimale fout kan dus ook bekomen worden als $x$ de oplossing van $\hat{R}x = c_1$\\
	
	\textbf{Complexiteit:} $\mathcal{O}(mn) + \mathcal{O}n^2$ indien QR-factorisatie bekend\\
	\textbf{Stabiliteit:} stabiel indien $A$ van volle rang
	
	
	
\end{document}